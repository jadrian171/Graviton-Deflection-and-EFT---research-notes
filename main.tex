% ****** Start of file aipsamp.tex ******
%
%   This file is part of the AIP files in the AIP distribution for REVTeX 4.
%   Version 4.1 of REVTeX, October 2009
%
%   Copyright (c) 2009 American Institute of Physics.
%
%   See the AIP README file for restrictions and more information.
%
% TeX'ing this file requires that you have AMS-LaTeX 2.0 installed
% as well as the rest of the prerequisites for REVTeX 4.1
%
% It also requires running BibTeX. The commands are as follows:
%
%  1)  latex  aipsamp
%  2)  bibtex aipsamp
%  3)  latex  aipsamp
%  4)  latex  aipsamp
%
% Use this file as a source of example code for your aip document.
% Use the file aiptemplate.tex as a template for your document.
\documentclass[%
 aip,
%jmp,%
%bmf,%
%sd,%
rsi,%
 amsmath,amssymb,
%preprint,%
 reprint,%
%author-year,%
%author-numerical,%
]{revtex4-1}

\usepackage{graphicx}% Include figure files
\usepackage{dcolumn}% Align table columns on decimal point
\usepackage{bm}% bold math
%\usepackage[mathlines]{lineno}% Enable numbering of text and display math
%\linenumbers\relax % Commence numbering lines

\begin{document}

\preprint{AIP/123-QED}

\title[Graviton Scattering and EFT - research notes]{Quantum corrections to the deflection angle of a gravitational wave by a massive scalar object}% Force line breaks with \\
%\thanks{Footnote to title of article.}

\author{JA Villanueva}
% \altaffiliation[Also at ]{Physics Department, XYZ University.}%Lines break automatically or can be forced with \\
\email{javillanueva@nip.upd.edu.ph}
\author{MFI Vega II}%
\affiliation{ 
National Institute of Physics, Univeresity of the Philippines Diliman%\\This line break forced with \textbackslash\textbackslash
}%

\date{\today}% It is always \today, today,
             %  but any date may be explicitly specified

\begin{abstract}
A completely valid quantum field theory for gravity is currently non-existent due to its non - renormalizablity. In spite of this difficulty, general relativity is treated as an ``effective" quantum field theory in the low energy regime, quite below the Planck mass, $M_P=\kappa^{-2}$, where divergences becomes renormalizable by a finite amount of parameters. In this paper, we use gravitational effective field theory to investigate the scattering of a gravitational wave, effectively treated as a single graviton particle, by a massive scalar field. We assume that the massive scalar field comes from a celestial object like a star. We compare it to the gravitational lensing of a photon traversing the same field. At lowest order, no difference in the deflection angle is observed. This result runs counter to the expectation that at one-loop, quantum corrections of order $\kappa^6$ will contribute a very small but significant change in the deflection angle.
\end{abstract}

\pacs{Valid PACS appear here}% PACS, the Physics and Astronomy
                             % Classification Scheme.
\keywords{Suggested keywords}%Use showkeys class option if keyword
                              %display desired
\maketitle

%\begin{quotation}
%The ``lead paragraph'' is encapsulated with the \LaTeX\ 
%\verb+quotation+ environment and is formatted as a single paragraph before the first section heading. 
%(The \verb+quotation+ environment reverts to its usual meaning after the first sectioning command.) 
%Note that numbered references are allowed in the lead paragraph.
%%
%The lead paragraph will only be found in an article being prepared for the journal \textit{Chaos}.
%\end{quotation}
\section{Introduction}
\hspace{\parindent}Of the four fundamental forces, gravity is the only one left without a completely valid quantum field theory \cite{zee2010quantum}. Its formulation is hindered by an infinitely increasing number of divergent parameters to renormalize, making the theory non-renormalizable. A non-renormalizable theory is not a valid quantum field theory since an infinite number of renormalized parameters to determine experimentally is not only non-sensible and non-predictive but also an issue in quantizing a field. Luckily, a non-renormalizable theory like general relativity for gravity is not useless for computing quantum effects in classical predictions provided only that the energy scale $E/M\ll 1$ holds wherein $M$ is a mass scale, typically taken to be the Planck Mass, $\kappa^{-2}=M_P=(32\pi G)^{-1}$ , but for some applications may be other masses in context like the mass of an electron or some scale. This means that one can quantize gravity to be an effective quantum field theory applicable only in a context of an energy scale well below the $M_P$ scale. For an effective field theory be valid, it is important for it to be only applied strictly in some variable scale such as the one described earlier. At this regime, we can take general relativity to be an ``effective" quantum field theory for gravity, making it a useful tool in studying quantum effects in gravitational fields \cite{donoghue1994general,bjerrum2003quantum}.
%

We use some of the typical conventions: (1) Minkowski metric signature is mostly plus $(-,+,+,+)$, (2) natural units are $\hbar=c=1$ ($G$ isnt suppressed in the results), (3) an on-shell particle satisfies the energy hyperboloid relation $p_\mu p^\mu\equiv p^2=-m^2$, (3) all $n$ external particles have outgoing momenta, $p_i$; with the conservation law stated as $\sum_{i}^{n}p_i=0$, and, (5) finally the Mandelstam invariants as $s_{ij}=-(p_i+p_j)^2$. 

\section{Overview of General Relativity as an Effective Field Theory}

\hspace{\parindent}The first objective in building an effective field theory is to determine what the action, hence the Lagrangian, looks like. Gravity will be quantized as a graviton field $h_{\mu\nu}$ (naturally a tensor field for a spin-2 particle), physically chosen to be a disturbance in the Minkowski metric, $g_{\mu\nu}=\eta_{\mu\nu}+h_{\mu\nu}$; which conceptually what weak-field gravity really is, a disturbance in flat space. 
To obtain a low-energy, local Lagrangian, we must construct the most general Lagrangian terms possible consistent with the symmetries of the theory (that is, we have coordinate invariance for general relativity). Using power counting, we neglect the terms that dominates in higher energies. To do this, the higher order terms must be suppressed by a very small constant.
It is claimed that in low energies general relativity must hold and therefore the terms left in a pure gravity Lagrangian must match with the general relativity Einstein-Hilbert (EH) action:

\begin{eqnarray}
	S&=&\int d^4 x\sqrt{-g}\left\lbrace\Lambda+\frac{2}{\kappa^2}R+c_1R^2+c_2R_{\mu\nu}R^{\mu\nu}+\ldots \right\rbrace \nonumber
	\\&\approx&\int d^4 x\sqrt{-g}\frac{2}{\kappa^2}R 
\end{eqnarray}

We see here that general relativity fits nicely as a gauge theory like QED, since if we took the leading term as the Ricci scalar curvature that has two derivatives in the field (which looks similarly a kinetic term), the Christoffel connection $\Gamma_{\mu\nu}^\sigma$, proportional to a single derivative in the metric, is naturally matched as a similar gauge term in a covariant derivative in a gauge theory, (which also, in gravity, the gauge theory covariant derivative will be the parallel transport covariant derivative).
 
Setting the coupling constant to match the EH action coupling, we have $\kappa^2=32\pi G$. Setting a particular gauge, one can obtain a propagator function for the graviton field by using the Ricci scalar curvature, $R$, as the kinetic term (the term quadratic in the field derivative). In the harmonic gauge $g^{\mu\nu}\Gamma^{\sigma}_{\mu\nu}=0$, the propagator in momentum space is:
\begin{equation}
	P_{\alpha\beta\mu\nu}(q)=i\frac{1}{2q^2}\Big( \eta_{\mu\alpha}\eta_{\nu\beta}+\eta_{\nu\alpha}\eta_{\mu\beta}-\eta_{\mu\nu}\eta_{\alpha\beta}\Big)
\end{equation}
For a massive scalar field, $\phi$ with mass $m$, the Lagrangian for the matter component of the action for the graviton to interact with, we can thereby complete our Feynman rules for the diagram calculations,
\begin{equation}
S_m=\int d^4 x\sqrt{-g}\left\lbrace g^{\mu\nu}\partial_\mu\phi\partial_\nu\phi - \frac{1}{2}m^2\phi^2\right\rbrace .
\end{equation}
Plugging in the $\sqrt{-g}=1+\eta^{\mu\nu}h_{\mu\nu}+\mathcal{O}(h^2)$, we can actively select the scalar-scalar-graviton $(\phi\phi h)$ vertex factor in the expansion, $V^{\mu\nu}(p_1,p_2)$, (in momentum space).
The kinetic term in the Lagrangian produces the propagator factor for a specific field as a function of the momentum transfer, $q$. For the massive scalar, this is equal to $D_\phi(q)$ (in momentum space).
\begin{eqnarray}
	V^{\mu\nu}(p_1,p_2)&=&i\frac{\kappa}{2}\Big(p_1^\mu p_2^\nu+ p_1^\nu p_2^\mu-\eta^{\mu\nu}(p_1\cdot p_2-m^2)\Big),\\ 	D_\phi(q)&=&i\frac{1}{q^2+m^2}  
\end{eqnarray}

Now, the Feynmam rules are complete, we can apply these to different GR problems provided we construct a relevant Feynman diagram, treating a system of interacting bodies like an event process in particle physics. 

\appendix

\section{Derivation of propagator and vertex factors fromthe action}

\end{document}
%
% ****** End of file aipsamp.tex ******